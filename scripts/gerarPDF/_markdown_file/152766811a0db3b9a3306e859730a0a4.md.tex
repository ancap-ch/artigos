\markdownRendererHeadingOne{Escrita, leitura e organização}\markdownRendererInterblockSeparator
{}\DECORAR{O}{} principal ponto da escrita é trazer ideias, informações ou conhecimento para o leitor. Este irá entender o texto melhor se estas ideias estiverem bem estruturadas, e irá ver e sentir esta estrutura muito melhor se a forma tipográfica refletir a estrutura lógica e semântica do contexto.\markdownRendererFootnote{Ver capítulo 2 \markdownRendererLink{daqui}{https://www.ime.usp.br/~reverbel/mac212-02/material/lshortBR.pdf}{https://www.ime.usp.br/~reverbel/mac212-02/material/lshortBR.pdf}{}.}\markdownRendererInterblockSeparator
{}O parágrafo é uma básica e importante unidade do texto que deve ser constituído por um pensamento coerente ou uma ideia. Suas quebras e continuações podem auxiliar o leitor a compreender o texto, uma vez que indicam uma alteração ou extensão de uma linha de pensamento. Desta forma, o leitor também lida com cadeias de pensamentos distintos e suas relações internas. Você também deveria estruturar seus parágrafos em um nível mais alto, colocando os mesmos em capítulos, seções, subseções\markdownRendererEllipsis{} para ajudar o leitor a encontrar"-se em seu trabalho.\markdownRendererInterblockSeparator
{}Uma estruturação ainda superior existe na relação entre textos distintos. Todos podemos nos beneficiar com a possibilidade da categorização, correlação (incluindo a ramificação) e do \markdownRendererEmphasis{feedback/and"-forth}, pois a especialização e nível crítico sobre um assunto é facilitada.\fimsubsec\markdownRendererInterblockSeparator
{}\markdownRendererHeadingTwo{Uma alternativa ao Facebook e aos blogs}\markdownRendererInterblockSeparator
{}O \markdownRendererEmphasis{Facebook} é muito popular e facilita a disseminação de textos curtos e médios. Porém suas discussões são limitadas em tipografia, espaço e estrutura. Como resultado, textos médios e longos exigem um maior esforço do leitor. Nesta plataforma, exceto aqueles mantidos por uma página, não é prático de se organizar e correlacionar textos distintos, até mesmo para serem usados como referência. Existe uma necessidade de ser ativamente respondido, caso contrário, a tendência é que seja abandonado. Esta é a filosofia da plataforma: não se ater muito aos textos antigos e ter um assunto que flui e é disperso em várias e incessantes mensagens.\markdownRendererInterblockSeparator
{}Nos blogs há uma grande fonte de textos que seriam simplesmente inviáveis se armazenados pelo e expostos no \markdownRendererEmphasis{Facebook}. Possibilitam uma referência consistente e a correlação facilitada entre os textos, apesar de terem um sistema de \markdownRendererEmphasis{Feedback} menos dinâmico do que o \markdownRendererEmphasis{Facebook} pois os textos estão dispersos em vários lugares. Isso não é uma desvantagem, é uma característica que é exigida pela total liberdade estrutural que o escritor tem em sua disposição.\markdownRendererInterblockSeparator
{}Este site (ancap.ch) faz uso da plataforma \markdownRendererEmphasis{\markdownRendererLink{discourse}{https://www.discourse.org/}{https://www.discourse.org/}{}} para a exposição e discussão em texto, que pode ser considerado um meio"-termo entre \markdownRendererEmphasis{Facebook} e blogs próprios. A plataforma \markdownRendererEmphasis{discourse} recomenda a escrita no formato \markdownRendererEmphasis{Markdown}\markdownRendererFootnote{\markdownRendererEmphasis{Markdown} é uma linguagem de marcação (\markdownRendererEmphasis{markup language}) que consiste em regras de fácil escrita, leitura e manutenção, no qual alguns símbolos são meta"-informações sobre o formato que o leitor verá. É parecido com as mensagens textuais dos emails.}, que resulta em uma apresentação de forma tão abrangente quanto a dos blogs. Além disto, facilita o \markdownRendererEmphasis{feedback} entre os usuários pela praticidade de criação de conteúdo e também pela organização de conteúdo através de classificação e correlação.\markdownRendererInterblockSeparator
{}\markdownRendererHeadingTwo{PDF}\markdownRendererInterblockSeparator
{}As plataformas discutidas até o momento são visualizadas pelo navegador (a partir de seu \markdownRendererEmphasis{HTML}) e, portanto, estão presas aos seus rápidos porém simples algoritmos de tipografia e formatação de texto. A exibição contínua do conteúdo exige uma atenção contínua do leitor. Não é fácil retornar ao último parágrafo depois que sua aba foi fechada, principalmente se alguns dias se passaram. Outra desvantagem desta visualização é a falta de praticidade para a leitura de notações de rodapé, que são normalmente utilizadas em artigos e livros, tornando"-se, na prática, um conteúdo ignorado e perdido. Desta forma, o leitor, já sabendo de tais fatos, tende a penalizar, de maneira antecipada, a leitura de textos maiores.\markdownRendererInterblockSeparator
{}Os arquivos em PDF não sofrem disto, podem ser lidos \markdownRendererEmphasis{offline} e quando o leitor desejar. Felizmente, podem ser produzidos e expostos de uma maneira prática.\markdownRendererInterblockSeparator
{}O programa \LaTeX\markdownRendererFootnote{Pronunciado como \textquotedblleft LÊiték\textquotedblright{}, foi criado por Leslie Lamport e pode usar o formatador TeX para produzir documentos de alta qualidade tipográfica.}, através do formatador \TeX\markdownRendererFootnote{Pronunciado \textquotedblleft Ték\textquotedblright{}, é um programa criado por Donald E. Knuth e lançado em 1982.} e do texto com marcações em \markdownRendererEmphasis{Markdown}, pode produzir arquivos em PDF com uma formatação de tipografia confortável e consistente, dado as configurações apropriadas. A intuitividade do \markdownRendererEmphasis{Markdown} limita o potencial de formatação do \LaTeX, mas considero que este arranjo complementa de forma simples e suficiente a visualização do conteúdo deste site.\markdownRendererInterblockSeparator
{}\markdownRendererHeadingOne{Editores}\markdownRendererInterblockSeparator
{}No computador, escreve"-se por um editor de texto. Os mais populares são o \markdownRendererEmphasis{MS Word} e o Bloco de Notas (que estou utilizando agora). Uma grande diferença entre ambos é a incapacidade deste de mudar a forma do texto visto pelo escritor (exceto uma mudança geral da fonte e seu tamanho). Um texto em \markdownRendererEmphasis{Markdown} conta exatamente com tal incapacidade: conta apenas com algumas marcações textuais que indicam tais formas (como texto em negrito ou sublinhado). Algumas pessoas preferem esta forma de texto (o Bloco de Notas é popular, lembra?) pois não há grandes preocupações nem distrações estéticas no momento em que se produz conteúdo; é como preferir se concentrar em um local mais silencioso ou numa temperatura mais confortável. Você pode encontrar informações de como utilizar os marcadores em \markdownRendererEmphasis{Markdown} \markdownRendererLink{aqui}{https://br.ancap.ch/t/26}{https://br.ancap.ch/t/26}{}. Não se preocupe, é bem óbvio.\markdownRendererInterblockSeparator
{}![ed1](\fontes/\caminho/editor1.PNG \textquotedblleft Editor padrão da plataforma discourse.\textquotedblright{})\markdownRendererInterblockSeparator
{}O editor da figura~\ref{fig:ed1} já vem embutido em toda distribuição padrão do \markdownRendererLink{sistema de discussão}{https://www.discourse.org/}{https://www.discourse.org/}{} utilizado por este site. Tal editor conta com a opção de pré"-visualizar a forma que seu texto terá para o leitor e suporta \markdownRendererEmphasis{boa parte} das marcações da linguagem \markdownRendererEmphasis{Markdown}. Até 2018 é esperado que o sistema seja atualizado para um mais simples e com mais funcionalidades. Como cada site pode mostrar um resultado diferente pro leitor a partir do mesmo arquivo \markdownRendererEmphasis{Markdown}, é aconselhável que você confira tal resultado ao publicar seu texto no site. Também é recomendável copiar o conteúdo do texto antes de postá"-lo pois você pode ter perdido conexão e dificilmente recuperará uma versão atualizada de seu texto (já aconteceu comigo).\markdownRendererInterblockSeparator
{}![ed2](\fontes/\caminho/editor2.PNG \textquotedblleft Editor HackMD.\textquotedblright{})\markdownRendererInterblockSeparator
{}O editor da figura~\ref{fig:ed2} é o \markdownRendererLink{HackMD}{https://hackmd.io/}{https://hackmd.io/}{} e também funciona pelo navegador. Ele suporta o modo de digitação \markdownRendererEmphasis{Emacs} e \markdownRendererEmphasis{Vim},\markdownRendererFootnote{\markdownRendererEmphasis{Emacs} e \markdownRendererEmphasis{Vim} são editores de texto que têm muitas combinações de teclas para ações no texto, como \textquotedblleft mover o cursor para o começo da linha\textquotedblright{} ou \textquotedblleft deletar a palavra sob o cursor\textquotedblright{}. Recomendo que procure sobre eles no Google.} que podem agilizar a escrita caso já se tenha experiência com estas tecnologias - são conhecimentos particularmente úteis para ajustar a formatação na publicação do texto. O editor também tem a opção de mostrar a pré"-visualização, além de poder ser escrito em colaboração com outras pessoas em tempo real, como um \markdownRendererEmphasis{Google Docs} (mas bem mais \textquotedblleft leve\textquotedblright{}). Vale lembrar que é sempre recomendável ter uma cópia salva dos arquivos \markdownRendererEmphasis{.md} localmente em seu computador, uma maneira simples de evitar grandes frustações.\markdownRendererInterblockSeparator
{}\markdownRendererHeadingOne{Publicação}\markdownRendererInterblockSeparator
{}\markdownRendererHeadingTwo{Online}\markdownRendererInterblockSeparator
{}Um navegador desenha na tela de acordo com as instruções no \markdownRendererEmphasis{HTML} do site. Um texto escrito com notações em \markdownRendererEmphasis{Markdown} é automaticamente transformado em \markdownRendererEmphasis{HTML} por este site. Porém tal transformatação possui algumas limitações.\markdownRendererInterblockSeparator
{}É incapaz de criar notas de rodapé com o direcionamento automático do usuário para o conteúdo da nota. Além disto, você deverá alterar seu texto para deixá"-lo nos conformes seguindo o modelo abaixo, escapando os colchetes:\markdownRendererInterblockSeparator
{}\markdownRendererInputVerbatim{./_markdown_file/ff0ece60d909a34da0ed088ddf1db506.verbatim}\markdownRendererInterblockSeparator
{}\markdownRendererHeadingTwo{PDF}\markdownRendererInterblockSeparator
{}Existem conversores de textos em \markdownRendererEmphasis{Markdown} para documentos em PDF. É recomendado uma produção pelo intermédio do \LaTeX para ter uma alta qualidade tipográfica. Para tal, existem conversores automáticos de textos em \markdownRendererEmphasis{Markdown} para \LaTeX, porém tais conversores não serão tratados.\fimsubsubsec\markdownRendererInterblockSeparator
{}\markdownRendererHeadingThree{Overleaf}\markdownRendererInterblockSeparator
{}O modelo usado oficialmente por este site pode ser encontrado \markdownRendererLink{aqui}{https://br.ancap.ch/t/modelos-latex-para-artigos/36}{https://br.ancap.ch/t/modelos-latex-para-artigos/36}{}. Para utilizá"-lo, basta seguir suas instruções. Você também pode utilizar como modelo a fonte \LaTeX \markdownRendererLink{deste próprio artigo}{https://www.overleaf.com/read/ssjjnsyqwsxt}{https://www.overleaf.com/read/ssjjnsyqwsxt}{}.\fimsubsubsec\markdownRendererInterblockSeparator
{}\markdownRendererHeadingThree{Dicas}\markdownRendererInterblockSeparator
{}É recomendável o aprendizado em \LaTeX caso queira utilizar fórmulas matemáticas e outros \markdownRendererEmphasis{features} próprio do \LaTeX. Também é recomendável para saber corrigir erros de compilação, que serão \markdownRendererStrongEmphasis{comuns} e serão corrigidos de acordo com a familiaridade com a ferramenta. No entanto, tentarei listar algumas dicas que anotei enquanto produzia alguns PDF's:\markdownRendererInterblockSeparator
{}\markdownRendererUlBegin
\markdownRendererUlItem Notas de rodapé:\markdownRendererInterblockSeparator
{}\markdownRendererUlBeginTight
\markdownRendererUlItem Não funcionam em cabeçalhos;\markdownRendererUlItemEnd 
\markdownRendererUlItem Não podem conter links. Para usá"-los, você deve escrever como no exemplo:\markdownRendererUlItemEnd 
\markdownRendererUlEndTight \markdownRendererUlItemEnd 
\markdownRendererUlEnd \markdownRendererInterblockSeparator
{}\markdownRendererInputVerbatim{./_markdown_file/539409bb67b1973ea222a04a3f87e1c9.verbatim}\markdownRendererInterblockSeparator
{}\markdownRendererUlBegin
\markdownRendererUlItem Escapamentos:\markdownRendererInterblockSeparator
{}\markdownRendererUlBeginTight
\markdownRendererUlItem Os \%{} indicam comentário no \LaTeX. Tome cuidado para escapá"-los para não perder conteúdo;\markdownRendererUlItemEnd 
\markdownRendererUlItem Você deve escapar os símbolos \%{}, \&{} e \hash{} com \textbackslash \%{}, \textbackslash \&{} e \textbackslash \hash{} respectivamente.\markdownRendererInterblockSeparator
{}\markdownRendererUlBeginTight
\markdownRendererUlItem Estes símbolos em URL não podem ser escapados. Prefira usar o \markdownRendererLink{encurtador do Google}{https://goo.gl/}{https://goo.gl/}{} para substituir tais links. De qualquer forma, a URL fica mais fácil de ser digitada caso seja lida de um documento impresso;\markdownRendererUlItemEnd 
\markdownRendererUlEndTight \markdownRendererUlItemEnd 
\markdownRendererUlEndTight \markdownRendererUlItemEnd 
\markdownRendererUlItem Imagens:\markdownRendererInterblockSeparator
{}\markdownRendererUlBeginTight
\markdownRendererUlItem Fonte por URL não funcionam. Devem ser enviadas para a pasta do projeto;\markdownRendererUlItemEnd 
\markdownRendererUlItem Legenda pode ser adicionada seguindo este modelo:\markdownRendererUlItemEnd 
\markdownRendererUlEndTight \markdownRendererUlItemEnd 
\markdownRendererUlEnd \markdownRendererInterblockSeparator
{}\markdownRendererInputVerbatim{./_markdown_file/72509d9fe73410713c957b3cc5c3d48f.verbatim}\markdownRendererInterblockSeparator
{}\markdownRendererUlBeginTight
\markdownRendererUlItem Sempre que for adicionar alguma URL em seu texto deste site (br.ancap.ch), você pode ignorar o título que aparece antes do final da URL, apenas o número do tópico (27, para este tópico) já basta.\markdownRendererUlItemEnd 
\markdownRendererUlEndTight \markdownRendererInterblockSeparator
{}Algumas destas dicas podem ser observadas \markdownRendererLink{neste video}{https://youtu.be/O-64IRaZlFs}{https://youtu.be/O-64IRaZlFs}{} de demonstração deste processo de geração de PDF.\relax