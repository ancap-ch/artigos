% Opções de tamanho de fonte: Huge, huge, LARGE, Large, large, normalsize, small
  
% Exemplo de título e subtítulo:
\def \titulo {Título do Artigo\LARGE{: Subtítulo do Artigo}} 
  
% Exemplo de dois autores:
\def \autor {Autor do Silva \& Autor Ferreira}

% Exemplo de dois autores com a mesma filiação e emails distintos:
\def \autor {
Autor~A\raisebox{5pt}{\normalsize$^{\ast 1}$} \& 
Autor~B\raisebox{5pt}{\normalsize$^{\dag 1}$}\\
{\normalfont\footnotesize 
$^1$ Redator, Instituto A\quad
$^\ast$\texttt{emailA@email.com}\quad
$^\dag$\texttt{emailB@email.com}}}
% Símbolos para notas de rodapé: \ast, \dag, \ddag, \S, \P, ...
  
% Exemplo de tradutor e revisor
% \def \tradutor {Fulano da Silva\newline\& \normalfont{Revisado por} \textsc{Beltrano da Silva}} 

% no caso nao-mobile, se quiser que novas seções, exceto a primeira, inicie em uma nova página:
\def \separarSecoes {} % nota: deve estar antes dos comandos de preâmbulo! (no começo do main.tex)

% no caso mobile, se quiser que o o quinto nível de hierarquia (nível-parágrafo) também inicie em uma nova página:
\def \clearparagraph {} % nota: deve estar antes dos comandos de preâmbulo! (no começo do main.tex)

% zera o rodapé a cada seção (requer, para não-mobile, que o separarSecoes esteja ativado)
\def \zeraRodape {} 

% -------------------------------------------------------------
% V. 1.2.2
% -------------------------------------------------------------
% 
% 1.2.2:
%  alterado tamanho das fontes do não-mobile, 72 CPL agora
%  alterando estrutura dos documentos. padronizar nome de capa, nomencluatura com _mob..
%  opção de incluir índice ou não está obsoleta. Não precisa mais.
%  add blackbox no final do documento
%  movido os comentários sobre o texto como um footnote no final do texto (arquivo comments.md)
%  movido as informações de data e "criador", e também do link da discussão, para um footnote no começo do texto
% 1.2.1:
%  fix de newline depois do índice
%  adição dos parágrafos como quebradores de página para mobile (optativo)
%  toc depth forçado em 2 para mobile
%  adicionado a funcionalidade que emula o clearpage das seções para formatos que não é mobile
%  capa agora é redimensionada e cortada automaticamente para o modo mobile
%  adicionado opção de zerar o rodapé a cada seção
% 1.2:
%  enviado para o gallery
%  seções, subseções e sub-subseções quebram a página automaticamente, exceto "as primeiras" (a primeira subseção, etc).
%  estrutura dos arquivos e main.tex modificados
%  tamanho da fonte para folha a4 modificada.
% 1.1.1:
%  fix de newline depois da capa
% 1.1.0:
%  novo feature: opção para gerar PDF para serem visualizados em uma tela de celular. 
%  fix: Informações do artigo ficaram mais centralizadas verticalmente.
%  improv: main.tex mais simples
% -------------------------------------------------------------