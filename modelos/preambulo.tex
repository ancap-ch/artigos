\def \caixa {\nopagebreak\hfill\textcolor{lightgray}{$\Box$}}

\ifdefined\mobile
      \documentclass[a6paper]{article}
    \usepackage[17pt]{extsizes}
    \usepackage[paperheight=175mm,paperwidth=105mm,left=0.28cm,top=.2cm,right=0.4cm,bottom=0.2cm,marginparwidth=0mm,marginparsep=0mm]{geometry}
 	\pagestyle{empty}
    \makeatletter
    \newif\if@sectionused \@sectionusedfalse
    \newif\if@subsectionused \@subsectionusedfalse
    \newif\if@subsubsectionused \@subsubsectionusedfalse
    \newif\if@paragraphused \@paragraphusedfalse
    \let\oldsection\section
    \let\oldsubsection\subsection
    \let\oldsubsubsection\subsubsection
    \let\oldparagraph\paragraph
    \renewcommand{\section}{
    \if@sectionused\clearpage\fi
    \@sectionusedtrue\@subsectionusedfalse\@subsubsectionusedfalse\@paragraphusedfalse
    \def \fimsec {\hfill\textcolor{lightgray}{$\Box$}}
    \def \fimsubsec {}
    \def \fimsubsubsec {}
    \ifdefined\zeraRodape\setcounter{footnote}{0}\fi
    \oldsection}
    \renewcommand{\subsection}{
    \if@subsectionused\clearpage\fi
    \@subsectionusedtrue\@subsubsectionusedfalse\@paragraphusedfalse
    \def \fimsubsec {\hfill\textcolor{lightgray}{$\Box$}}
    \def \fimsubsubsec {}
    \oldsubsection}
    \renewcommand{\subsubsection}{
    \if@subsubsectionused\clearpage\fi
    \@subsubsectionusedtrue\@paragraphusedfalse
    \def \fimsubsubsec {\hfill\textcolor{lightgray}{$\Box$}}
    \oldsubsubsection}
    %\renewcommand{\paragraph}{\if@paragraphused\clearpage\fi\@paragraphusedtrue\oldparagraph}
    \makeatother
\else
  \documentclass[a4paper]{article}
  \usepackage[17pt]{extsizes}
  \usepackage[top=1.185in, bottom=1.185in, left=.4in, right=.4in]{geometry}
  %\ifdefined\separarSecoes
  	    \makeatletter
    \newif\if@sectionused \@sectionusedfalse
    \let\oldsection\section
    \renewcommand{\section}{
    \ifdefined\separarSecoes
	    \if@sectionused\clearpage\fi
	    \@sectionusedtrue
    	\def \fimsec {\hfill\textcolor{lightgray}{$\Box$}}
    \fi
    \ifdefined\zeraRodape\setcounter{footnote}{0}\fi
    \oldsection
    }
    \makeatother
  %\fi
\fi


\pdfminorversion=6
\pdfcompresslevel=9
\pdfobjcompresslevel=3

\usepackage[utf8]{inputenc}
\usepackage{libertine}
\usepackage{libertinust1math}
\usepackage[T1]{fontenc}

%\usepackage{geometry}
%\usepackage[left=0cm,top=0cm,right=0cm,bottom=0cm,marginparwidth=0mm,marginparsep=0mm,margin=0cm]{geometry}


% para o markdown
\usepackage[footnotes,definitionLists,hashEnumerators,smartEllipses, hybrid]{markdown}
  \ifdefined\mobile
    \markdownSetup{rendererPrototypes={
      link = {\href{#2}{#1}}
    }}
  \else
    \markdownSetup{rendererPrototypes={
    link = {\href{#2}{#1}\footnoteB{\href{#2}{{\ttfamily\scriptsize\relax$ \langle $#2$ \rangle $}}}}
    }}
  \fi
\newcommand{\hash}{\#}


\usepackage[brazil]{babel}
\usepackage[babel=true,kerning=true]{microtype}
%
\babelhyphenation[portuguese]{li-vre-mer-ca-do}
\babelhyphenation[portuguese]{an-ti-fluo-re-ta-ci-o-ni-sta}
\babelhyphenation[portuguese]{an-ti-fluo-re-ta-ci-o-ni-sta}
\babelhyphenation[portuguese]{so-ci-al-de-mo-cra-tas}
\babelhyphenation[portuguese]{es-pe-re-pe-lo-re-sul-ta-do}
\babelhyphenation[portuguese]{bem-su-ce-di-das}
\babelhyphenation[portuguese]{axio-má-ti-co-de-du-ti-vos}
\babelhyphenation[portuguese]{ló-gi-co-pra-xeo-ló-gi-ca}
\babelhyphenation[portuguese]{pseu-do-pro-ble-ma}
\babelhyphenation[portuguese]{pri-mei-ro-usu-á-rio-pri-mei-ro-do-no}
\babelhyphenation[portuguese]{con-tra-ar-gu-men-to}
\babelhyphenation[portuguese]{em-pi-ri-sta-in-tui-ti-vis-ta}
\babelhyphenation[portuguese]{não-cog-ni-ti-vis-mo}
\babelhyphenation[portuguese]{não-cog-ni-ti-vis-tas}
\babelhyphenation[portuguese]{não-ra-cio-nal}
\babelhyphenation[portuguese]{não-de-du-ti-va}
\babelhyphenation[portuguese]{ló-gi-co-se-mân-ti-co}
\babelhyphenation[portuguese]{auto-pre-mi-a-do}
\babelhyphenation[portuguese]{Mises-Roth-bard}
\babelhyphenation[portuguese]{pré-ca-pi-ta-lis-tas}
\babelhyphenation[portuguese]{não-apro-pri-a-do-res}
\babelhyphenation[portuguese]{não-pro-du-to-res}
\babelhyphenation[portuguese]{não-pou-pa-do-res}
\babelhyphenation[portuguese]{mi-sesiana-ro-th-bardiana}
\babelhyphenation[portuguese]{di-fe-ren-ças-ch-ave}
\babelhyphenation[portuguese]{não-ex-plo-ra-to-ria-men-te}
\babelhyphenation[portuguese]{pro-pri-e-tá-rio-pro-du-tor}
\babelhyphenation[portuguese]{não-ho-mes-tea-ders}
\babelhyphenation[portuguese]{não-con-tra-tan-tes}
\babelhyphenation[portuguese]{não-pro-du-ti-vas}
\babelhyphenation[portuguese]{não-con-tra-tuais}
\babelhyphenation[portuguese]{não-pro-du-ti-va}
\babelhyphenation[portuguese]{não-con-tra-tual}
\babelhyphenation[portuguese]{mal-di-re-cio-nar}
\babelhyphenation[portuguese]{não-ho-mes-tea-der-pro-du-tor-pou-pa-dor-con-tra-tan-te}
\babelhyphenation[portuguese]{ho-mes-tea-der-pro-dutor-pou-pa-dor-con-tra-tan-te}
\babelhyphenation[portuguese]{so-ma-ze-ro}
\babelhyphenation[portuguese]{mais-va-lia}
\babelhyphenation[portuguese]{es-ta-dos-ban-cos-em-pre-sas}
\babelhyphenation[portuguese]{não-marx-ista}
\babelhyphenation[portuguese]{não-ca-pi-ta-li-sta}
\babelhyphenation[portuguese]{va-lor-tra-ba-lho}
\babelhyphenation[portuguese]{anti-es-ta-tis-tas}
\babelhyphenation[portuguese]{auto-pro-cla-ma-do}
\babelhyphenation[portuguese]{Böhm-Ba-werk}
\babelhyphenation[portuguese]{So-cia-lism-Com-mu-nism}
\babelhyphenation[portuguese]{pa-drão-ou-ro}
\babelhyphenation[portuguese]{his-tó-ri-cas-so-cio-ló-gi-cas}
\babelhyphenation[portuguese]{Anti-Sta-tist}
\babelhyphenation[portuguese]{auto-im-pos-tas}
\babelhyphenation[portuguese]{neo-marx-ista}
\babelhyphenation[portuguese]{aus-tría-ca-li-ber-tá-ria}
\babelhyphenation[portuguese]{Hans-Her-mann}
\babelhyphenation[portuguese]{li-be-ral-li-ber-tá-ria}
\babelhyphenation[portuguese]{anar-co-in-di-vi-dua-lis-ta}
\babelhyphenation[portuguese]{pró-mer-ca-do}
\babelhyphenation[portuguese]{anti-es-ta-do}
\babelhyphenation[portuguese]{não-li-ber-tá-rios}
\babelhyphenation[portuguese]{meta-in-for-ma-ções}
\babelhyphenation[portuguese]{fura-gre-ves}
\babelhyphenation[portuguese]{pseudo-hu-ma-ni-tá-rio}
\babelhyphenation[portuguese]{su-pri-mi-lo}
\babelhyphenation[portuguese]{Es-ta-dos-Na-ções}
\babelhyphenation[portuguese]{me-lhor-por-que-mais-uni-ver-sal-dos-di-nhei-ros}
\babelhyphenation[portuguese]{di-nhe-iro-mer-ca-do-ria}
\babelhyphenation[portuguese]{re-cém-con-ce-di-do}
\babelhyphenation[portuguese]{anti-frau-de}
\babelhyphenation[portuguese]{di-nhei-ro-pa-drão-mer-ca-do-ria}
\babelhyphenation[portuguese]{pa-drão-fi-du-ci-á-rio}
\babelhyphenation[portuguese]{cir-cui-to-fe-cha-do}
\babelhyphenation[portuguese]{ex-pan-são-re-tra-ção}
\babelhyphenation[portuguese]{es-ta-do-ban-cos}
\babelhyphenation[portuguese]{pa-péis-moeda}
\babelhyphenation[portuguese]{ces-sar-fo-go}
\babelhyphenation[portuguese]{pré-mo-der-na}
\babelhyphenation[portuguese]{so-cia-lis-tas-au-to-ri-tá-rios}
\babelhyphenation[portuguese]{pa-péis-mo-e-da}
\babelhyphenation[portuguese]{an-ti-im-pe-ria-lis-tas}
\babelhyphenation[portuguese]{pa-drão-dó-lar}
\babelhyphenation[portuguese]{em-pre-sa-rial-ban-cá-ria-es-ta-tal}
\babelhyphenation[portuguese]{pré-marx-ista}
\babelhyphenation[portuguese]{Saint-Si-mon}
\babelhyphenation[portuguese]{ban-cos-ne-gó-cios}
\babelhyphenation[portuguese]{so-cial-de-mo-cra-ta}
\babelhyphenation[portuguese]{Ex-vi-ce-pre-si-den-te}
\babelhyphenation[portuguese]{não-con-ser-va-do-res}
\babelhyphenation[portuguese]{qua-se-hayek-iano}
\babelhyphenation[portuguese]{ter-se-ia}
\babelhyphenation[portuguese]{po-lí-ti-co-ideo-ló-gi-co}
\babelhyphenation[portuguese]{li-vro-tex-to}
\babelhyphenation[portuguese]{em-pi-ri-sta-falsi-fi-ca-cio-nis-ta}
\babelhyphenation[portuguese]{mo-nis-mo-de-con-ve-rsa}
\babelhyphenation[portuguese]{en-cer-ra-dor-de-con-ver-sa}
\babelhyphenation[portuguese]{fi-ló-so-fos-reis}
\babelhyphenation[portuguese]{analí-ti-co-empí-ri-co-nor-ma-ti-vo}
\babelhyphenation[portuguese]{em-ci-ma-do-mu-ris-mo}
\babelhyphenation[portuguese]{não-se-com-pro-me-ta-com-nada}
\babelhyphenation[portuguese]{não-ra-cio-nais}
\babelhyphenation[portuguese]{não-in-fe-ren-cial-men-te}
\babelhyphenation[portuguese]{ter-em-vis-ta}
\babelhyphenation[portuguese]{celidônia-menor}
\babelhyphenation[portuguese]{qua-se-in-fe-ren-cial}
\babelhyphenation[portuguese]{so-bre-in-te-lec-tua-i-li-za}
\babelhyphenation[portuguese]{so-bre-in-te-lec-tua-li-zar}



\usepackage{tocloft} % para mudar o titulo do indice
\usepackage{etoc} % para saber se tem indice ou não
% https://tex.stackexchange.com/questions/94961/how-to-check-in-latex-whether-the-table-of-content-is-empty-or-not-before-added
\etocchecksemptiness % do not display empty local table of contents
\etocnotocifnotoc % do not display empty global table of contents



% usado para a capa...
\usepackage{tikz}
\usetikzlibrary{positioning}
\usepackage{varwidth}
\pgfdeclarelayer{bg}    % declare background layer
\pgfdeclarelayer{front} 
\pgfsetlayers{bg,main,front}  % set the order of the layers (main is the standard layer)
\usepackage[hidelinks,
pdftex,
  pdfauthor={\mautor},
  pdftitle={\mtitulo},
  pdfsubject={Ancap/artigos},
  pdfkeywords={},
  pdfproducer={source in markdown},
  pdfcreator={Thiago Machado da Silva}
]{hyperref}
%\usepackage{calc}
\usetikzlibrary{calc}
\usepackage{graphicx}
\usepackage{calc}
\usepackage{ifthen}


\DeclareUrlCommand\Hurl{%
  \def\UrlLeft{\langle}%
  \def\UrlRight{\rangle}%
}
\renewcommand*{\UrlFont}{\ttfamily\scriptsize\relax}

%\usepackage{psvectorian}
%\let\clipbox\relax
\usepackage{pgfornament}


% usado par ao índice
\addto\captionsbrazil{
  \renewcommand{\contentsname}
    {}% A frase que aparece no lugar de "Conteúdo" no índice
}

% usado para limitar o tamanho das imagens
\usepackage[export]{adjustbox}

% usado para esticar e cortar imagens
% https://tex.stackexchange.com/questions/60918/how-to-scale-and-then-trim-an-image
\newlength{\oH}
\newlength{\oW}
\newlength{\rH}
\newlength{\rW}
\newlength{\cH}
\newlength{\cW}
\newcommand\ClipImage[3]{% width, height, image
\settototalheight{\oH}{\includegraphics{#3}}%
\settowidth{\oW}{\includegraphics{#3}}%
\setlength{\rH}{\oH * \ratio{#1}{\oW}}%
\setlength{\rW}{\oW * \ratio{#2}{\oH}}%
\ifthenelse{\lengthtest{\rH < #2}}{%
    \setlength{\cW}{(\rW-#1)*\ratio{\oH}{#2}}%
    \adjincludegraphics[height=#2,clip,trim=0 0 \cW{} 0]{#3}%
}{%
    \setlength{\cH}{(\rH-#2)*\ratio{\oW}{#1}}%
    \adjincludegraphics[width=#1,clip,trim=0 \cH{} 0 0]{#3}%
}%
}


%\def \titulo {Título do Artigo} 
%\def \autor {} 
%\def \tradutor {} 
%\def \url {} 
%\def \CriadorDestePDF {}

% usado para fonte decorativa
\usepackage{lettrine}

%\usepackage{GoudyIn}
%\renewcommand{\LettrineFontHook}{\GoudyInfamily{}}
%\newcommand{\DECORAR}[3][]{\lettrine[lines=3,loversize=.115,#1]{#2}{#3}}

\usepackage{Zallman}
\renewcommand{\LettrineFontHook}{\Zallmanfamily{}}
\newcommand{\DECORAR}[3][]{\lettrine[lines=3,loversize=.115,#1]{#2}{#3}}


\ifdefined\mobile
  \ifdefined\mobileDepth
    \setcounter{tocdepth}{3}
  \else
    \setcounter{tocdepth}{2}
  \fi
\fi

% para listar o link da discussão e a data:
\makeatletter
\def\blfootnote{\xdef\@thefnmark{$ \sim $}\@footnotetext}
\makeatother
% https://tex.stackexchange.com/questions/250221/supressing-the-footnote-number

\makeatletter
\newcommand*{\myfnsymbolsingle}[1]{%
  \ensuremath{%
    \ifcase#1% 0
    \or % 1
      *%   
    \or % 2
      \dagger
    \or % 3  
      \ddagger
    \or % 4   
      \mathsection
    \or % 5
      \mathparagraph
    \else % >= 6
      \@ctrerr  
    \fi
  }%   
}   
\makeatother
\newcommand*{\myfnsymbol}[1]{%
  \myfnsymbolsingle{\value{#1}}%
}
\makeatletter
\newcommand*{\greekfnsymbolsingle}[1]{%
  \ensuremath{%
  \ifcase#1\or\alpha\or\beta\or\gamma\or\delta\or\varepsilon
    \or\zeta\or\eta\or\theta\or\iota\or\kappa\or\lambda
    \or\mu\or\nu\or\xi\or o\or\pi\or\varrho\or\sigma
    \or\tau\or\upsilon\or\phi\or\chi\or\psi\or\omega
    \else\@ctrerr\fi
  }%   
}   
\makeatother
\newcommand*{\greekfnsymbol}[1]{%
  \greekfnsymbolsingle{\value{#1}}%
}

\ifdefined\mobile
  \usepackage[hang,flushmargin,multiple]{footmisc}
  \newcommand{\nota}[1]{[#1]}
  \let\oldfootnote\footnote
  \renewcommand{\footnote}[1]{\oldfootnote{#1\caixa}}
\else
  \usepackage[flushmargin,multiple]{footmisc}
  \usepackage{bigfoot}
  \DeclareNewFootnote{default}
  \DeclareNewFootnote[para]{C}
  \DeclareNewFootnote[para]{B}[fnsymbol]
  \MakeSortedPerPage{footnoteB}
  \MakeSortedPerPage{footnoteC}
  \usepackage{alphalph}
  \newalphalph{\myfnsymbolmult}[mult]{\myfnsymbolsingle}{}
  \newalphalph{\greekfnsymbolmult}[mult]{\greekfnsymbolsingle}{}
  
  \renewcommand*{\thefootnoteB}{%
    \myfnsymbolmult{\value{footnoteB}}%
  }
  \renewcommand*{\thefootnoteC}{%
    \greekfnsymbolmult{\value{footnoteC}}%
  }
  \newcommand{\nota}[1]{\footnoteC{#1}}
  \let\oldfootnote\footnote
  \renewcommand{\footnote}[1]{\oldfootnote{#1\caixa}}
\fi


\def \fimsec {}
\def \fimsubsec {}
\def \fimsubsubsec {}

%\usepackage{pgffor}
\usepackage{arrayjob}
\usepackage{calc}

\newcommand{\linhatitulo}[1]{%
\checktitulos(#1)\ifemptydata\else%
  \begingroup\resizebox{.878\linewidth}{!}{\textsc {\cachedata}}\endgroup%
  \checktitulos({\the\numexpr #1+1})\ifemptydata\else\newline\\\fi%
\fi}

% \newcommand{\linhatitulo}[1]{
%   \checktitulos(#1)
%   \ifemptydata\else
%   \begingroup\resizebox{.878\linewidth}{!}{\textsc {\cachedata}}\endgroup
%   %\begingroup\resizebox{.878\linewidth}{!}{\textsc {\cachedata}}\endgroup
%   \checktitulos({\the\numexpr #1+1})\ifemptydata\else\newline\\\fi
%   \fi}
