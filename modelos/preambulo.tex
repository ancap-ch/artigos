\ifdefined\mobile
      \documentclass[a6paper]{article}
    \usepackage[17pt]{extsizes}
    \usepackage[paperheight=175mm,paperwidth=105mm,left=0.28cm,top=.2cm,right=0.4cm,bottom=0.2cm,marginparwidth=0mm,marginparsep=0mm]{geometry}
 	\pagestyle{empty}
    \makeatletter
    \newif\if@sectionused \@sectionusedfalse
    \newif\if@subsectionused \@subsectionusedfalse
    \newif\if@subsubsectionused \@subsubsectionusedfalse
    \newif\if@paragraphused \@paragraphusedfalse
    \let\oldsection\section
    \let\oldsubsection\subsection
    \let\oldsubsubsection\subsubsection
    \let\oldparagraph\paragraph
    \renewcommand{\section}{
    \if@sectionused\clearpage\fi
    \@sectionusedtrue\@subsectionusedfalse\@subsubsectionusedfalse\@paragraphusedfalse
    \def \fimsec {\hfill\textcolor{lightgray}{$\Box$}}
    \def \fimsubsec {}
    \def \fimsubsubsec {}
    \ifdefined\zeraRodape\setcounter{footnote}{0}\fi
    \oldsection}
    \renewcommand{\subsection}{
    \if@subsectionused\clearpage\fi
    \@subsectionusedtrue\@subsubsectionusedfalse\@paragraphusedfalse
    \def \fimsubsec {\hfill\textcolor{lightgray}{$\Box$}}
    \def \fimsubsubsec {}
    \oldsubsection}
    \renewcommand{\subsubsection}{
    \if@subsubsectionused\clearpage\fi
    \@subsubsectionusedtrue\@paragraphusedfalse
    \def \fimsubsubsec {\hfill\textcolor{lightgray}{$\Box$}}
    \oldsubsubsection}
    %\renewcommand{\paragraph}{\if@paragraphused\clearpage\fi\@paragraphusedtrue\oldparagraph}
    \makeatother
\else
  \documentclass[a4paper]{article}
  \usepackage[14pt]{extsizes}
  \usepackage[top=1.185in, bottom=1.185in, left=1in, right=1in]{geometry}
  \ifdefined\separarSecoes
  	    \makeatletter
    \newif\if@sectionused \@sectionusedfalse
    \let\oldsection\section
    \renewcommand{\section}{
    \ifdefined\separarSecoes
	    \if@sectionused\clearpage\fi
	    \@sectionusedtrue
    	\def \fimsec {\hfill\textcolor{lightgray}{$\Box$}}
    \fi
    \ifdefined\zeraRodape\setcounter{footnote}{0}\fi
    \oldsection
    }
    \makeatother
  \fi
\fi

\pdfminorversion=6
\pdfcompresslevel=9
\pdfobjcompresslevel=3

\usepackage[utf8]{inputenc}
\usepackage{libertine}
\usepackage{libertinust1math}
\usepackage[T1]{fontenc}

%\usepackage{geometry}
%\usepackage[left=0cm,top=0cm,right=0cm,bottom=0cm,marginparwidth=0mm,marginparsep=0mm,margin=0cm]{geometry}

% para o markdown
\usepackage[footnotes,definitionLists,hashEnumerators,smartEllipses, hybrid]{markdown}
  \ifdefined\mobile
    \markdownSetup{rendererPrototypes={
      link = {\href{#2}{#1}}
    }}
  \fi
\newcommand{\hash}{\#}

\usepackage[portuguese]{babel}
\usepackage{microtype}

\usepackage{tocloft} % para mudar o titulo do indice
\usepackage{etoc} % para saber se tem indice ou não
% https://tex.stackexchange.com/questions/94961/how-to-check-in-latex-whether-the-table-of-content-is-empty-or-not-before-added
\etocchecksemptiness % do not display empty local table of contents
\etocnotocifnotoc % do not display empty global table of contents

% usado para a capa...
\usepackage{tikz}
\usetikzlibrary{positioning}
\usepackage{varwidth}
\pgfdeclarelayer{bg}    % declare background layer
\pgfdeclarelayer{front} 
\pgfsetlayers{bg,main,front}  % set the order of the layers (main is the standard layer)
\usepackage[hidelinks]{hyperref}
%\usepackage{calc}
\usetikzlibrary{calc}
\usepackage{graphicx}
\usepackage{calc}
\usepackage{ifthen}


% usado par ao índice
\addto\captionsportuguese{
  \renewcommand{\contentsname}
    {}% A frase que aparece no lugar de "Conteúdo" no índice
}

% usado para limitar o tamanho das imagens
\usepackage[export]{adjustbox}

% usado para esticar e cortar imagens
% https://tex.stackexchange.com/questions/60918/how-to-scale-and-then-trim-an-image
\newlength{\oH}
\newlength{\oW}
\newlength{\rH}
\newlength{\rW}
\newlength{\cH}
\newlength{\cW}
\newcommand\ClipImage[3]{% width, height, image
\settototalheight{\oH}{\includegraphics{#3}}%
\settowidth{\oW}{\includegraphics{#3}}%
\setlength{\rH}{\oH * \ratio{#1}{\oW}}%
\setlength{\rW}{\oW * \ratio{#2}{\oH}}%
\ifthenelse{\lengthtest{\rH < #2}}{%
    \setlength{\cW}{(\rW-#1)*\ratio{\oH}{#2}}%
    \adjincludegraphics[height=#2,clip,trim=0 0 \cW{} 0]{#3}%
}{%
    \setlength{\cH}{(\rH-#2)*\ratio{\oW}{#1}}%
    \adjincludegraphics[width=#1,clip,trim=0 \cH{} 0 0]{#3}%
}%
}


%\def \titulo {Título do Artigo} 
%\def \autor {} 
%\def \tradutor {} 
%\def \url {} 
%\def \CriadorDestePDF {}

% usado para fonte decorativa
\usepackage{lettrine}
\usepackage{GoudyIn}
\renewcommand{\LettrineFontHook}{\GoudyInfamily{}}
\newcommand{\DECORAR}[3][]{\lettrine[lines=3,loversize=.115,#1]{#2}{#3}}

\ifdefined\mobile
  \setcounter{tocdepth}{2}
\fi

% para listar o link da discussão e a data:
\makeatletter
\def\blfootnote{\xdef\@thefnmark{}\@footnotetext}
\makeatother
% https://tex.stackexchange.com/questions/250221/supressing-the-footnote-number

\usepackage[hang,flushmargin]{footmisc}
\usepackage{color}

\let\oldfootnote\footnote
% adapted from https://tex.stackexchange.com/questions/315804/strip-last-character-from-parameter-if-it-is-s
%\def\striplastS#1{\striplastSa{#1}#1\end .\end\eend}
%\def\striplastSa#1#2.\end#3\eend{\ifx\end#3\end#1\else#2\fi}
\renewcommand{\footnote}[1]{\oldfootnote{#1\nopagebreak\hfill\textcolor{lightgray}{$\Box$}}}

\ifdefined\mobile
  \def \fimsec {\hfill\textcolor{lightgray}{$\Box$}}
  \def \fimsubsec {\hfill\textcolor{lightgray}{$\Box$}}
  \def \fimsubsubsec {\hfill\textcolor{lightgray}{$\Box$}}
\else
  \ifdefined\separarSecoes
    \def \fimsec {\hfill\textcolor{lightgray}{$\Box$}}
    \def \fimsubsec {}
    \def \fimsubsubsec {}
  \else
    \def \fimsec {}
    \def \fimsubsec {}
    \def \fimsubsubsec {}
  \fi
\fi