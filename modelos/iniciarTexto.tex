\ifdefined\nomeSite
\else
  \def \nomeSite {AncapChannel}
\fi
\ifdefined\URL
  \ifdefined\CriadorDestePDF
    \ifdefined\mobile
      \def \poster {Discussão em \href{\URL}{\nomeSite}, por \textsc{\CriadorDestePDF}.}
    \else
      \def \poster {Discussão em \href{\URL}{\nomeSite}\footnoteB{\href{\URL}{\URL}}, por \textsc{\CriadorDestePDF}.}
    \fi
  \else
    \ifdefined\mobile
      \def \poster {Discussão em \href{\URL}{\nomeSite}.}
    \else
      \def \poster {Discussão em \href{\URL}{\nomeSite}\footnoteB{\url{\URL}}.}
    \fi
  \fi
\else
  \ifdefined\CriadorDestePDF
    \def \poster {Arquivo gerado por \textsc{\CriadorDestePDF}.}
  \else
    \def \poster {}
  \fi
\fi
\ifdefined\incluirCapa
  \noindent\begin{tikzpicture}[remember picture]
  
	\begin{pgfonlayer}{front}
  
  \coordinate (titulo_c) at (0,0);
  
  \ifdefined\titulo 
    \node [name=titulo_t,
    anchor=north west,
    fill=white,
    text = black,
    align=center %,
    %text width=.8\paperwidth
    %minimum width=\paperwidth
    ,opacity=0
    ] at (titulo_c) {
      \ifdefined\mobile
     	\begin{varwidth}{\textwidth}\LARGE{\textsc{\titulo}}\end{varwidth}
      \else
     	\begin{varwidth}{\textwidth}\LARGE{\textsc{\titulo}}\end{varwidth}
      \fi
    };
    \coordinate [below=0.01\paperheight] (autor_c) at (titulo_t.south west);
  \else
  	\coordinate (titulo_t) at (titulo_c);
    \coordinate [below=0.01\paperheight] (autor_c) at (titulo_t);
  \fi
  
  \ifdefined\autor
    \node [name=autor_t,
    ,anchor=north west
    ,fill=white
    ,text = black
    ,align=center
    ,opacity=0
    ] at (autor_c) {
      \ifdefined\mobile
     	\begin{varwidth}{\textwidth}\large{\textsc{\autor}}\end{varwidth}};
      \else
     	\begin{varwidth}{\textwidth}\Large{\textsc{\autor}}\end{varwidth}};
      \fi
    \coordinate [below=0.01\paperheight] (tradutor_c) at (autor_t.south west);
  \else
  	\coordinate (autor_t) at (autor_c);
    \coordinate [below=0.01\paperheight] (tradutor_c) at (autor_t);
  \fi
  
  \ifdefined\tradutor
    \node [name=tradutor_t,
    ,anchor=north west
    ,fill=white
    ,text = black
    ,align=center
    ,opacity=0
    ] at (tradutor_c) {
      \ifdefined\mobile
     	\begin{varwidth}{\textwidth}{Tradução: \textsc{\tradutor}}\end{varwidth}};
      \else
     	\begin{varwidth}{\textwidth}{Traduzido por \textsc{\tradutor}}\end{varwidth}};
      \fi
  	\coordinate [below=0.01\paperheight] (poster_c) at (tradutor_t.south west);
  \else
  	\coordinate (tradutor_t) at (tradutor_c);
  	\coordinate [below=0.01\paperheight] (poster_c) at (tradutor_t);
  \fi
  
  \coordinate [below=0.07\paperheight] (branco_c) at (tradutor_t.south west);

    \end{pgfonlayer}
  \end{tikzpicture}

\noindent\begin{tikzpicture}[remember picture,overlay]

      \coordinate (nw) at (current page.north west);
      \ifdefined\capa
       \noindent\begin{pgfonlayer}{bg}    % select the background layer
            \noindent\node[inner sep=0, anchor=north west,name=capa] at (current page.north west) {
              \ifdefined\mobile
                \noindent\ClipImage{\paperwidth}{\paperheight}{\capa}
            	%\noindent\includegraphics[min width=\paperwidth,min height=\paperheight,keepaspectratio]{\capa}
              \else
            	\noindent\includegraphics[width=\paperwidth,keepaspectratio]{\capa}
              \fi
            };
            
              \ifdefined\mobile
              \else
            	\fill [anchor=west, fill=white] (-11,0) rectangle (current page.south east);
              \fi
            
            
        \end{pgfonlayer}
      \fi

    
	\begin{pgfonlayer}{front}    % select the background layer
  %\coordinate [ajuste] (nome) at (lugar)
  

  	  \coordinate (zero) at (0,0);
  	  \coordinate (titulo_c2) at ($ (titulo_t.north west) + (zero) - (titulo_c) + (0,1) $);
      
      
        \ifdefined\mobile
         \def \tituloL {.5cm}
         \def \tituloU {-.5cm}
      	\else
         \def \tituloL {1.0in}
         \def \tituloU {-.060\paperheight}
     	\fi
        
      \ifdefined\titulo
        \node [name=titulo_t2,
        anchor=north west,
        fill=white,
        text = black,
        %align=center %,
        %text width=.8\paperwidth
        %minimum width=\paperwidth
        ,fill opacity=0.85
        ,text opacity=1
        %,left=-6.1cm 
        ] at ($ (nw) + (\tituloL,\tituloU) $) {
        \ifdefined\mobile
     		\begin{varwidth}{\textwidth}\LARGE{\textsc{\titulo}}\end{varwidth}
      	\else
     		\begin{varwidth}{\textwidth}\LARGE{\textsc{\titulo}}\end{varwidth}
     	\fi
        };
        
        \coordinate [below=0.01\paperheight] (autor_c2) at (titulo_t2.south west);
      \else
  	  	\coordinate (titulo_t2) at (titulo_c2);
        \coordinate [below=0.01\paperheight] (autor_c2) at (titulo_t2);
      \fi

      \ifdefined\autor
        \node [name=autor_t2,
        ,anchor=north west
        ,fill=white
        ,text = black
        ,align=center
        ,fill opacity=0.85
        ,text opacity=1
        ] at (autor_c2) {
        \ifdefined\mobile
          \begin{varwidth}{\textwidth}\large{\textsc{\autor}}\end{varwidth}};
        \else
          \begin{varwidth}{\textwidth}\Large{\textsc{\autor}}\end{varwidth}};
        \fi
        \coordinate [below=0.01\paperheight] (tradutor_c2) at (autor_t2.south west);
      \else
  	  	\coordinate (autor_t2) at (autor_c2);
        \coordinate [below=0.01\paperheight] (tradutor_c2) at (autor_t2);
      \fi

      \ifdefined\tradutor
        \node [name=tradutor_t2,
        ,anchor=north west
        ,fill=white
        ,text = black
        ,align=center
        ,fill opacity=0.85
        ,text opacity=1
        ] at (tradutor_c2) {
        \ifdefined\mobile
          \begin{varwidth}{\textwidth}{Tradução: \textsc{\tradutor}}\end{varwidth}};
        \else
          \begin{varwidth}{\textwidth}{Traduzido por \textsc{\tradutor}}\end{varwidth}};
        \fi
      	\coordinate [below=0.01\paperheight] (poster_c2) at (tradutor_t2.south west);
      \else
  	  	\coordinate (tradutor_t2) at (tradutor_c2);
        \coordinate [below=0.01\paperheight] (poster_c2) at (tradutor_t2);
      \fi
      
    \end{pgfonlayer}
     
  \end{tikzpicture}

  \ifdefined\mobile
    \clearpage
  \else
  \fi
\else
\fi % a próxima linha já deve ser o TOC
\ifdefined\mobile
  \let\oldfootnoterule\footnoterule
  \renewcommand*\footnoterule{}
\else\ifdefined\comecaseparado
  \let\oldfootnoterule\defaultfootnoterule
  \renewcommand*\defaultfootnoterule{}
\fi\fi
\blfootnote{\today. \poster\caixa}
%\renewcommand\cftaftertoctitle{\par\noindent\hrulefill\par\vskip-0.65em}
\renewcommand\cftaftertoctitle{\par\vskip-3.15em}
\tableofcontents
\etocifwasempty 
{
  \ifdefined\mobile
    \let\footnoterule\oldfootnoterule
  \else
    \newline%\newline % se tiver linha em branco acima, da pau
    \vskip-3.15em
  \fi
}
{
  %\psvectorian{47} %
  \ifdefined\mobile
    \def \hr{\leavevmode\leaders\hrule height 0.556pt\hfill\kern0pt}
    \noindent\raisebox{.194pt}{\pgfornament[width=.068\textwidth]{14}}\noindent\hr
  \else
    \def \hr{\leavevmode\leaders\hrule height 0.542pt\hfill\kern0pt}
    \noindent\raisebox{.194pt}{\pgfornament[width=.035\textwidth]{14}}\noindent\hr
  \fi
  %\noindent\hr\raisebox{.300pt}{\pgfornament[width=.050\textwidth]{84}}\noindent\hr\hr\hr\hr\hr
  %\noindent\hr\raisebox{.580pt}{\pgfornament[width=.111\textwidth]{89}}\noindent\hr\hr
  %\noindent\hrulefill
  \ifdefined\mobile
    \thispagestyle{empty}
    \clearpage
    \let\footnoterule\oldfootnoterule
  \else
    \ifdefined\comecaseparado
      \clearpage
      \let\defaultfootnoterule\oldfootnoterule
    \else
      %\newline
    \fi
  \fi
}
\frenchspacing 
\markdownSetup{rendererPrototypes={
  image = {\begin{figure}[hbt!]
    \centering
    \includegraphics[max width=\textwidth]{#3}%
    \ifx\empty#4\empty\else
    \caption{#4}%
    \fi
    \label{fig:#1}%
    \end{figure}}
}}
\markdownInput{\fontes/\caminho/artigo_tmp.md}
