  \ifdefined\incluirCapa
	\ifdefined\mobile
   \def \tituloU {-.7173cm}
\else
   %\def \tituloU {-.060\paperheight}
   \def \tituloU {-.021\paperheight}
\fi

\noindent\begin{tikzpicture}[remember picture]

  \begin{pgfonlayer}{front}
  
    \coordinate (titulo_c) at (0,0);
    
    \ifdefined\titulos
      \node [name=titulo_t,anchor=north,fill=white,text = black,align=center,opacity=0] 
        at ($ (titulo_c) + (0,\tituloU) $) {
      %text width=.8\paperwidth%minimum width=\paperwidth
        \linhatitulo{1}\linhatitulo{2}\linhatitulo{3}\linhatitulo{4}\linhatitulo{5}\linhatitulo{6}
      };
      \coordinate [below=0.01\paperheight] (autor_c) at (titulo_t.south);
    \else
      \coordinate (titulo_t) at (titulo_c);
      \coordinate [below=0.01\paperheight] (autor_c) at (titulo_t);
    \fi
  
    \ifdefined\autor
      \node [name=autor_t,anchor=north,fill=white,text = black,align=center,opacity=0] at (autor_c) {
        \ifdefined\mobile
          \begin{varwidth}{.97\textwidth}\large{\textsc{\autor}}\end{varwidth}};
        \else
          \begin{varwidth}{.98\textwidth}\Large{\textsc{\autor}}\end{varwidth}};
        \fi
      \coordinate [below=0.01\paperheight] (tradutor_c) at (autor_t.south);
    \else
      \coordinate (autor_t) at (autor_c);
      \coordinate [below=0.01\paperheight] (tradutor_c) at (autor_t);
    \fi
  
    \ifdefined\tradutor
      \node [name=tradutor_t,anchor=north,fill=white,text = black,align=center,opacity=0] at (tradutor_c) {
        \ifdefined\mobile
          \begin{varwidth}{.97\textwidth}{Tradução: \textsc{\tradutor}}\end{varwidth}};
        \else
          \begin{varwidth}{.98\textwidth}{Traduzido por \textsc{\tradutor}}\end{varwidth}};
        \fi
      \coordinate [below=0.01\paperheight] (poster_c) at (tradutor_t.south);
    \else
      \coordinate (tradutor_t) at (tradutor_c);
      \coordinate [below=0.01\paperheight] (poster_c) at (tradutor_t);
    \fi
  
    \coordinate [below=0.07\paperheight] (branco_c) at (tradutor_t.south);

  \end{pgfonlayer}
\end{tikzpicture}

\noindent\begin{tikzpicture}[remember picture,overlay]

  \coordinate (nw) at (current page.north west);
  \coordinate (nn) at (current page.north);

  \ifdefined\capa
    \noindent\begin{pgfonlayer}{bg}    % select the background layer
      \noindent\node[inner sep=0, anchor=north west,name=capa] at (current page.north west) {
        \ifdefined\mobile
          \noindent\ClipImage{\paperwidth}{\paperheight}{\capa}
          %\noindent\includegraphics[min width=\paperwidth,min height=\paperheight,keepaspectratio]{\capa}
        \else
          \noindent\includegraphics[width=\paperwidth,keepaspectratio]{\capa}
        \fi
      };
            
      \ifdefined\mobile\else
        \fill [anchor=west, fill=white] (-11,.5) rectangle (current page.south east);
      \fi
    \end{pgfonlayer}
  \fi

  \begin{pgfonlayer}{front}    % select the background layer
    \coordinate (zero) at (0,0);
    \coordinate (titulo_c2) at ($ (titulo_t.north) + (zero) - (titulo_c) + (0,1) $);
        
    \ifdefined\titulos
      \node [name=titulo_t2,anchor=north,fill=white,text = black,align=center,fill opacity=0.87,text opacity=1] 
        at ($ (nn) + (0,\tituloU) $) {
      %text width=.8\paperwidth %minimum width=\paperwidth
        \linhatitulo{1}%
        \linhatitulo{2}%
        \linhatitulo{3}%
        \linhatitulo{4}%
        \linhatitulo{5}%
        \linhatitulo{6}%
      };
      \coordinate [below=0.01\paperheight] (autor_c2) at (titulo_t2.south);
    \else
      \coordinate (titulo_t2) at (titulo_c2);
      \coordinate [below=0.01\paperheight] (autor_c2) at (titulo_t2);
    \fi

    \ifdefined\autor
      \node [name=autor_t2,anchor=north,fill=white,text = black,align=center,fill opacity=0.87,text opacity=1] 
        at (autor_c2) {
      \ifdefined\mobile
        \begin{varwidth}{.97\textwidth}\large{\textsc{\autor}}\end{varwidth}};
      \else
        \begin{varwidth}{.98\textwidth}\Large{\textsc{\autor}}\end{varwidth}};
      \fi
      \coordinate [below=0.01\paperheight] (tradutor_c2) at (autor_t2.south);
    \else
      \coordinate (autor_t2) at (autor_c2);
      \coordinate [below=0.01\paperheight] (tradutor_c2) at (autor_t2);
    \fi

    \ifdefined\tradutor
      \node [name=tradutor_t2,anchor=north,fill=white,text = black,align=center,fill opacity=0.87,text opacity=1] 
        at (tradutor_c2) {
      \ifdefined\mobile
        \begin{varwidth}{.97\textwidth}{Tradução: \textsc{\tradutor}}\end{varwidth}};
      \else
        \begin{varwidth}{.98\textwidth}{Traduzido por \textsc{\tradutor}}\end{varwidth}};
      \fi
      \coordinate [below=0.01\paperheight] (poster_c2) at (tradutor_t2.south);
    \else
      \coordinate (tradutor_t2) at (tradutor_c2);
      \coordinate [below=0.01\paperheight] (poster_c2) at (tradutor_t2);
    \fi
      
  \end{pgfonlayer}
\end{tikzpicture}


    \ifdefined\mobile
      \clearpage
    \else
    \fi
  \else
  \fi % a próxima linha já deve ser o TOC
  \renewcommand\cftaftertoctitle{\par\noindent\hrulefill\par\vskip-0.65em}
  \tableofcontents
  \etocifwasempty 
  {
    \ifdefined\mobile
    \else
     \newline % se tiver linha em branco acima, da pau
    \fi
  }
  {
  \noindent\hrulefill
  \ifdefined\mobile
    \thispagestyle{empty}
    \clearpage
  \else
    \newline
  \fi
  }

\frenchspacing 
\markdownSetup{rendererPrototypes={
  image = {\begin{figure}[hbt!]
    \centering
    \includegraphics[max width=\textwidth]{#3}%
    \ifx\empty#4\empty\else
    \caption{#4}%
    \fi
    \label{fig:#1}%
    \end{figure}}
}}

  \ifdefined\URL
    \ifdefined\CriadorDestePDF
      \def \poster {\today. Discussão em: <\URL>, por \textsc{\CriadorDestePDF}}
    \else
      \def \poster {\today. Discussão em: <\URL>}
    \fi
  \else
    \ifdefined\CriadorDestePDF
      \def \poster {\today. Arquivo gerado por \textsc{\CriadorDestePDF}}
    \else
      \def \poster {\today}
    \fi
  \fi

\blfootnote{$ ^\sim $\poster}

\markdownInput{\fontes/\caminho/artigo.md}
